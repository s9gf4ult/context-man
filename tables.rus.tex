\title{Таблицы}

\subject{Общие принципы}

Таблицы в \c верстаются с помощью 3х пар команд.


\startdescr{code:} \hfill
\startTEX
\bTABLE
\bTR
\bTD \tex{bTABLE} / \tex{eTABLE} \eTD
\bTD Начало / конец таблицы \eTD
\eTR

\bTR
\bTD \tex{bTR} / \tex{eTR} \eTD
\bTD Начало / конец строки в таблице \eTD
\eTR

\bTR
\bTD \tex{bTD} / \tex{eTD} \eTD
\bTD Начало / конец колонки в строке \eTD
\eTR
\eTABLE
\stopTEX
\stopdescr

\startdescr{result:} \hfill
\startCODERESULT
\bTABLE
\bTR
\bTD \tex{bTABLE} / \tex{eTABLE} \eTD
\bTD Начало / конец таблицы \eTD
\eTR

\bTR
\bTD \tex{bTR} / \tex{eTR} \eTD
\bTD Начало / конец строки в таблице \eTD
\eTR

\bTR
\bTD \tex{bTD} / \tex{eTD} \eTD
\bTD Начало / конец колонки в строке \eTD
\eTR
\eTABLE
\stopCODERESULT
\stopdescr

Пример выше - одновременно пример, и пояснение по командам. Общее правило для
верстки таблиц - это вложенность. На самом верхнем уровне находится таблица. Все
тело таблицы должно быть между \tex{bTABLE} и \tex{eTABLE}. Само тело состоит из
строк, содержание каждой строки таблицы располагается между \tex{bTR} и
\tex{eTR}. Далее, каждая строка содержит в себе столбцы, содержимое которых
располагается между \tex{bTD} и \tex{eTD}.

\vbox{
Вот еще пример таблицы:

\startdescr{code:} \hfill
\startTEX
\bTABLE
\bTR
\bTD 1 \eTD
\bTD 2 \eTD
\bTD 3 \eTD
\bTD 4 \eTD
\bTD 5 \eTD
\bTD 6 \eTD
\eTR

\bTR
\bTD один   \eTD
\bTD два    \eTD
\bTD три    \eTD
\bTD четыре \eTD
\bTD пять   \eTD
\bTD шесть  \eTD
\eTR
\eTABLE
\stopTEX
\stopdescr

\startdescr{result:} \hfill
\startCODERESULT
\bTABLE
\bTR
\bTD 1 \eTD
\bTD 2 \eTD
\bTD 3 \eTD
\bTD 4 \eTD
\bTD 5 \eTD
\bTD 6 \eTD
\eTR

\bTR
\bTD один   \eTD
\bTD два    \eTD
\bTD три    \eTD
\bTD четыре \eTD
\bTD пять   \eTD
\bTD шесть  \eTD
\eTR
\eTABLE
\stopCODERESULT
\stopdescr
}

\subject{Объединенные ячейки}

Объединенные ячейки создаются с помощью опциональных
