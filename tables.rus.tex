\title{Таблицы}

\subject{Общие принципы}

Таблицы в \c верстаются с помощью 3х пар команд.


\startdescr{code:} \hfill
\startTEX
\bTABLE
\bTR
\bTD \tex{bTABLE} / \tex{eTABLE} \eTD
\bTD Начало / конец таблицы \eTD
\eTR

\bTR
\bTD \tex{bTR} / \tex{eTR} \eTD
\bTD Начало / конец строки в таблице \eTD
\eTR

\bTR
\bTD \tex{bTD} / \tex{eTD} \eTD
\bTD Начало / конец колонки в строке \eTD
\eTR
\eTABLE
\stopTEX
\stopdescr

\startdescr{result:} \hfill
\startCODERESULT
\bTABLE
\bTR
\bTD \tex{bTABLE} / \tex{eTABLE} \eTD
\bTD Начало / конец таблицы \eTD
\eTR

\bTR
\bTD \tex{bTR} / \tex{eTR} \eTD
\bTD Начало / конец строки в таблице \eTD
\eTR

\bTR
\bTD \tex{bTD} / \tex{eTD} \eTD
\bTD Начало / конец колонки в строке \eTD
\eTR
\eTABLE
\stopCODERESULT
\stopdescr

Пример выше - одновременно пример, и пояснение по командам. Общее правило для
верстки таблиц - это вложенность. На самом верхнем уровне находится таблица. Все
тело таблицы должно быть между \tex{bTABLE} и \tex{eTABLE}. Само тело состоит из
строк, содержание каждой строки таблицы располагается между \tex{bTR} и
\tex{eTR}. Далее, каждая строка содержит в себе столбцы, содержимое которых
располагается между \tex{bTD} и \tex{eTD}.

\vbox{
Вот еще пример таблицы:

\startdescr{code:} \hfill
\startTEX
\bTABLE
\bTR
\bTD 1 \eTD
\bTD 2 \eTD
\bTD 3 \eTD
\bTD 4 \eTD
\bTD 5 \eTD
\bTD 6 \eTD
\eTR

\bTR
\bTD один   \eTD
\bTD два    \eTD
\bTD три    \eTD
\bTD четыре \eTD
\bTD пять   \eTD
\bTD шесть  \eTD
\eTR
\eTABLE
\stopTEX
\stopdescr

\startdescr{result:} \hfill
\startCODERESULT
\bTABLE
\bTR
\bTD 1 \eTD
\bTD 2 \eTD
\bTD 3 \eTD
\bTD 4 \eTD
\bTD 5 \eTD
\bTD 6 \eTD
\eTR

\bTR
\bTD один   \eTD
\bTD два    \eTD
\bTD три    \eTD
\bTD четыре \eTD
\bTD пять   \eTD
\bTD шесть  \eTD
\eTR
\eTABLE
\stopCODERESULT
\stopdescr
}

\subject{Объединенные ячейки}

Объединенные ячейки создаются с помощью опциональных аргументов для команды
\tex{bTD}.

\placetable[here]{Аргументы \tex{bTD} для создания объединенных ячеек}{
\bTABLE

\bTR
\bTD nc \eTD
\bTD Количество колонок, занимаемое ячейкой. \eTD
\eTR

\bTR
\bTD nr \eTD
\bTD Количество строк, занимаемое ячейкой. \eTD
\eTR

\eTABLE}

И сразу пример:

\vbox{
\startdescr{code:} \hfill
\startTEX
\bTABLE
\bTR
\bTD[nc=2] Эта ячейка занимает две колонки \eTD
\eTR

\bTR
\bTD Простая ячейка \eTD
\bTD Тоже простая \eTD
\eTR
\eTABLE
\stopTEX
\stopdescr
}

\vbox{
\startdescr{result:} \hfill
\startCODERESULT
\bTABLE
\bTR
\bTD[nc=2] Эта ячейка занимает две колонки \eTD
\eTR

\bTR
\bTD Простая ячейка \eTD
\bTD Тоже простая \eTD
\eTR
\eTABLE
\stopCODERESULT
\stopdescr
}

Обратите внимане, что первая строка в таблице содержит лишь одну ячейку, но эта
колонка занимает сразу две ячейки. Вторая же строка содержит две ячейки и они
верстаются как обычно.

\vbox{
Теперь пример с многострочной ячейкой:

\startdescr{code:} \hfill
\startTEX
\bTABLE
\bTR
\bTD[nr=2] Ячейка занимает две стркои \eTD
\bTD обычная ячейка \eTD
\eTR

\bTR
\bTD тоже обычная \eTD
\eTR
\eTABLE
\stopTEX
\stopdescr
}

\vbox{
\startdescr{result:} \hfill
\startCODERESULT
\bTABLE
\bTR
\bTD[nr=2] Ячейка занимает две стркои \eTD
\bTD обычная ячейка \eTD
\eTR

\bTR
\bTD тоже обычная \eTD
\eTR
\eTABLE
\stopCODERESULT
\stopdescr
}

Тут все немного менее очевидно. Для начала, мы видим, что в первой строке
указано две ячейки, тогда как во второй только одна. Дело в том, что
многостроковая ячейка занимает, в данном случае, две строки: свою, первую
строку, и следующую. Таким образом, во второй строке первая колонка уже занята
многострочной ячейкой из предыдущей строки.

Вот еще пример:

\vbox{
\startdescr{code:} \hfill
\startTEX
\bTABLE
\bTR
\bTD[nr=3] Ячейка занимает три стркои, свою и две нижних.
Первая колонка в следующих двух строках уже занята этой
ячейкой. \eTD
\bTD обычная ячейка \eTD
\eTR

\bTR
\bTD тоже обычная \eTD
\eTR

\bTR
\bTD опять простая \eTD
\eTR
\eTABLE
\stopTEX
\stopdescr
}

\vbox{
\startdescr{result:} \hfill
\startCODERESULT
\bTABLE
\bTR
\bTD[nr=3] Ячейка занимает три стркои, свою и две нижних.
Первая колонка в следующих двух строках уже занята этой
ячейкой. \eTD
\bTD обычная ячейка \eTD
\eTR

\bTR
\bTD тоже обычная \eTD
\eTR

\bTR
\bTD опять простая \eTD
\eTR
\eTABLE
\stopCODERESULT
\stopdescr
}

Более сложный пример:

\vbox{
\startdescr{code:} \hfill
\startTEX
\bTABLE
\bTR
\bTD[nr=3] Ячейка занимает три стркои, свою и две нижних.
Первая колонка в следующих двух строках уже занята этой
ячейкой. \eTD
\bTD обычная ячейка \eTD
\bTD[nr=2] Ячейка занимает две строки, свою и следующую. \eTD
\eTR

\bTR
\bTD тоже обычная \eTD
\eTR

\bTR
\bTD опять простая \eTD
\bTD на этой строке верхняя ячейка уже не распространяется. \eTD
\eTR
\eTABLE
\stopTEX
\stopdescr
}

\vbox{
\startdescr{result:} \hfill
\startCODERESULT
\bTABLE
\bTR
\bTD[nr=3] Ячейка занимает три стркои, свою и две нижних.
Первая колонка в следующих двух строках уже занята этой
ячейкой. \eTD
\bTD обычная ячейка \eTD
\bTD[nr=2] Ячейка занимает две строки, свою и следующую. \eTD
\eTR

\bTR
\bTD тоже обычная \eTD
\eTR

\bTR
\bTD опять простая \eTD
\bTD на этой строке верхняя ячейка уже не распространяется. \eTD
\eTR
\eTABLE
\stopCODERESULT
\stopdescr
}
