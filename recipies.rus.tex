\title{Различные рецепты}

\subject{Как автоматически расположить картинки по горизонтали}

Иногда надо расположить картинки так, чтобы высота была одинаковая, а суммарная
ширина картинок равнялась ширине текста (ну или заданному значению). Этого можно
достичь следующим образом:

\vbox{
\startdescr{code:} \hfill
\startTEX
\scale[width=\textwidth]{
  \externalfigure[wide.jpg][height=1cm]%
  \externalfigure[square.jpg][height=1cm]
}
\stopTEX
\stopdescr
}

\vbox{
\startdescr{result:} \hfill

\scale[width=\textwidth]{
  \externalfigure[wide.jpg][height=1cm]%
  \externalfigure[square.jpg][height=1cm]
}

\stopdescr
}

Обратите внимание на знак коментария после первой команды
\tex{externalfigure}. Это для того, чтобы между картинками не вставлялся пробел,
ведь перевод строки рассматривается \c как пробельный символ, и потому в строке
оказывается пробел между картинками, если убрать знак коментраия, то картинки
будут прижиматься не вплотную.

\vbox{
\startdescr{code:} \hfill
\startTEX
\scale[width=\textwidth]{
  \externalfigure[wide.jpg][height=1cm]
  \externalfigure[square.jpg][height=1cm]
}
\stopTEX
\stopdescr
}

\vbox{
\startdescr{result:} \hfill

\scale[width=\textwidth]{
  \externalfigure[wide.jpg][height=1cm]
  \externalfigure[square.jpg][height=1cm]
}

\stopdescr
}

В принципе, можно было бы просто записать две команды \tex{externalfigure} в
одну строку без пробелов, но это не очень удобно при редактировании исходного
текста. Знак кометрания коментирует не просто окончание строки, но и сам символ перевода строки, таким образом висящий в конце знак коментрария позволяет располагать содержимое следующей строки без пробелов.

\vbox{
\startdescr{code:} \hfill
\startTEX
\scale[width=\textwidth]{
  \externalfigure[wide.jpg][height=1cm]%
  \externalfigure[square.jpg][height=1cm]%
  \externalfigure[example.jpg][height=1cm]%
  \externalfigure[wide.jpg][height=1cm]
}
\stopTEX
\stopdescr
}

\vbox{
\startdescr{result:} \hfill

\scale[width=\textwidth]{
  \externalfigure[wide.jpg][height=1cm]%
  \externalfigure[square.jpg][height=1cm]%
  \externalfigure[example.jpg][height=1cm]%
  \externalfigure[wide.jpg][height=1cm]
}

\stopdescr
}

Обратите внимание, как в примере выше картинки стали уже. Тут происходит
следующее: сначала картинки располагаются в строчку с одинаковой высотой (1cm),
после чего вся строка картинок масштабируется под нужную ширину. Высота картинок
при этом изменяется пропорционально.

Можно изменить соотношение высот картинок в строке просто поменяв изначальную
высоту картинок пропорционально:

\vbox{
\startdescr{code:} \hfill
\startTEX
\scale[width=\textwidth]{
  \externalfigure[wide.jpg][height=1cm]%
  \externalfigure[wide.jpg][height=2cm]
}
\stopTEX
\stopdescr
}

\vbox{
  \startdescr{result:} \hfill

  \scale[width=\textwidth]{
    \externalfigure[wide.jpg][height=1cm]%
    \externalfigure[wide.jpg][height=2cm]
  }

  \stopdescr
}

Обратите внимане, как изменились пропорции высот, с тем же успехом, можно
вставлять картинки с пропорциональной шириной.

\vbox{
\startdescr{code:} \hfill
\startTEX
\scale[width=\textwidth]{
  \externalfigure[wide.jpg][width=2cm]%
  \externalfigure[square.jpg][width=1cm]
}
\stopTEX
\stopdescr
}

\vbox{
  \startdescr{result:} \hfill

  \scale[width=\textwidth]{
    \externalfigure[wide.jpg][width=2cm]%
    \externalfigure[square.jpg][width=1cm]
  }

  \stopdescr
}

В примере выше левая картинка в два раза шире, чем
правая. Изначальные абсолютные размеры картинок не играют
особой роли, важны лишь пропорции, ибо картинки все равно
масштабируются под заданную ширину.

\subject{Как расположить картинку и текст одним блоком}

Просто используйте таблицы. В \c таблицы - это универсальный
инструмент для верстки. Если выключить рамки таблиц, то
можно с помощью них располагать элементы выравненными
относительно друг друга.


\vbox{
\startdescr{code:} \hfill
\startTEX
\bTABLE
\setupTABLE[column][1][width=\dimexpr(\textwidth/3)]
\setupTABLE[align={lohi,width,stretch,tolerant},frame=off]
\bTR
\bTD \externalfigure[square.jpg][factor=fit] \eTD
\bTD \input knuth \eTD
\eTR
\eTABLE
\stopTEX
\stopdescr
}

\vbox{
\startdescr{result:} \hfill

\bTABLE
\setupTABLE[column][1][width=\dimexpr(\textwidth/3)]
\setupTABLE[align={lohi,width,stretch,tolerant},frame=off]
\bTR
\bTD \externalfigure[square.jpg][factor=fit] \eTD
\bTD \input knuth \eTD
\eTR
\eTABLE

\stopdescr
}

Таким образом, можно формировать нечто вроде следующей страницы:

\page

{\dontleavehmode
  \vfill

\bTABLE
\setupTABLE[column][1][width=\dimexpr(\textwidth/2)]
\setupTABLE[frame=off,align={width,stretch,tolerant,lohi}]
\bTR
\bTD \externalfigure[square.jpg][factor=fit] \eTD
\bTD
\startitemize
\item Самый мощный и производительный источник плазменной
  резки в линейке Powermax.

\item Рабочий режим и надежность этого аппарата отвечает
  самым высоким требованиям.

\item Powermax 105 режет металл толщиной до 38 мм (с
  кромки).

\item {\bf Номинальная толщина механизированного прожига –
  до 22 мм.}

\stopitemize
\eTD
\eTR
\eTABLE

\blank[1cm]
\bTABLE
\setupTABLE[column][2][width=\dimexpr(\textwidth/2)]
\setupTABLE[frame=off,align={width,stretch,tolerant,lohi}]
\bTR
\bTD
\startitemize
\item Может быть использован в качестве источника для ручной
  плазменной резки, а так же в составе различных систем и
  установок автоматизированной плазменной резки с ЧПУ.

\item Аппарат оснащен защитой и автоматическим выключением
  при критическом износе расходных деталей.

\item Технология Smart Sense™ автоматически регулирует
  давление газа в соответствии с режимом резки и длиной
  провода резака для достижения оптимальной
  производительности.

\stopitemize
\eTD
\bTD \externalfigure[161.png][factor=fit] \eTD
\eTR
\eTABLE

\bigskip
Оснащенный последними высокотехнологическими разработками,
такими как технология Smart Sense™ для автоматической
регулировки давления газа, Powermax 65 позволяет сделать
больше, чем когда-либо прежде. Четыре шаблона резака
Duramax™ повышают гибкость применения и позволяют операторам
выбирать нужный для работы инструмент.
\vfill
}

\page

А вот и исходник этой страницы:

\vbox{
\startdescr{code:} \hfill
\startTEX
\page

{\dontleavehmode
  \vfill

\bTABLE
\setupTABLE[column][1][width=\dimexpr(\textwidth/2)]
\setupTABLE[frame=off,align={width,stretch,tolerant,lohi}]
\bTR
\bTD \externalfigure[square.jpg][factor=fit] \eTD
\bTD
\startitemize
\item Самый мощный и производительный источник плазменной
  резки в линейке Powermax.

\item Рабочий режим и надежность этого аппарата отвечает
  самым высоким требованиям.

\item Powermax 105 режет металл толщиной до 38 мм (с
  кромки).

\item {\bf Номинальная толщина механизированного прожига –
  до 22 мм.}

\stopitemize
\eTD
\eTR
\eTABLE
\stopTEX
\stopdescr
}

\vbox{
\startdescr{code:} \hfill
\startTEX
\blank[1cm]
\bTABLE
\setupTABLE[column][2][width=\dimexpr(\textwidth/2)]
\setupTABLE[frame=off,align={width,stretch,tolerant,lohi}]
\bTR
\bTD
\startitemize

\item Может быть использован в качестве источника для ручной
  плазменной резки, а так же в составе различных систем и
  установок автоматизированной плазменной резки с ЧПУ.

\item Аппарат оснащен защитой и автоматическим выключением
  при критическом износе расходных деталей.

\item Технология Smart Sense™ автоматически регулирует
  давление газа в соответствии с режимом резки и длиной
  провода резака для достижения оптимальной
  производительности.

\stopitemize
\eTD
\bTD \externalfigure[161.png][factor=fit] \eTD
\eTR
\eTABLE

\bigskip
Оснащенный последними высокотехнологическими разработками,
такими как технология Smart Sense™ для автоматической
регулировки давления газа, Powermax 65 позволяет сделать
больше, чем когда-либо прежде. Четыре шаблона резака
Duramax™ повышают гибкость применения и позволяют операторам
выбирать нужный для работы инструмент.
\vfill
}

\page
\stopTEX
\stopdescr
}
