\chapter{Общие правила}

Весь текст в \c разбит на параграфы. Параграфы при верстке отделаются
одной (и более) пустой строкой.

\startdescr{code:} \hfill
\startTEX
Параграф один.

Параграф два.


Параграф три.
\stopTEX
\stopdescr

\startdescr{result:} \hfill
\startCODERESULT
Параграф один.

Параграф два.


Параграф три.
\stopCODERESULT
\stopdescr

В примере выше три параграфа. \c игнорирует идущие подряд пустые строки, так
что, отбить расстояние энтерами как в верде не получится. В данном случае
расстояние между параграфами после верстки будет одинаковое.

Также, в \c есть команды. Все команды начинаются с \tex{} (бэкслеш). Чтобы
напечатать сам бэкслеш нужно ввести команду ``\tex{tex}'' вот так

\vbox{
\startdescr{code:} \hfill
\startTEX
\tex{}
\stopTEX
\stopdescr
}

\vbox{
\startdescr{result:} \hfill
\startCODERESULT
\tex{}
\stopCODERESULT
\stopdescr
}

Кроме бэкслеша есть и другие специальные символы. С бэкслеша начинаются команды,
а с \% начинаются коментарии. Все что идет после знака коментария
игонорируется \c. Чтобы напечатать знак \%, а не коментарий
нужно \quote{оквочить} символ \% бэкслешем.

\vbox{
\startdescr{code:} \hfill
\startTEX
% Все, что идет после знака коментария игнорируется.
% Этот текст не будет виден после компиляции
Около 95\% опрошенных считают, что небо синего цвета.
\% все, что идет после знака процента видно
\stopTEX
\stopdescr
}

\vbox{
\startdescr{result:} \hfill
\startCODERESULT
% Все, что идет после знака коментария игнорируется.
% Этот текст не будет виден после компиляции
Около 95\% опрошенных считают, что небо синего цвета.
\% все, что идет после знака процента видно
\stopCODERESULT
\stopdescr
}

Кроме знака \% в \c есть и другие специальные символы.

\placetable[here]{Специальные символы}{
\bTABLE
\setupTABLE[align={flushleft,lohi}]
\setupTABLE[column][3][width=4cm]
\bTR
\bTD Символ \eTD
\bTD Назначение \eTD
\bTD Как напечатать сам сивол \eTD
\eTR

\bTR
\bTD \% \eTD
\bTD Символ коментария. Все, что идет после него до конца строки игнорируется. \eTD
\bTD \tex{\%} \eTD
\eTR

\bTR
\bTD \tex{} \eTD
\bTD С этого символа начинаются команды \eTD
\bTD \tex{tex\{\}} \eTD
\eTR

\bTR
\bTD \$ \eTD
\bTD С этого символа начинаются математические формулы, им же и заканчиваются \eTD
\bTD \tex{\$} \eTD
\eTR

\bTR
\bTD \{ \} \eTD
\bTD Фигурные скобки используются для группирования текста и/или команд.
Также, в фигурных скобках передаются аргументы для команд. \eTD
\bTD \tex{\{} \tex{\}} \eTD
\eTR

\bTR
\bTD \# \eTD
\bTD Используется в формировании новых команд, обычно не нужен при простой верстке \eTD
\bTD \tex{\#} \eTD
\eTR

\eTABLE}

По сути, когда вы набираете \quote{\tex} с точки зрения \c ничем не отличается
от \quote{\tex{placefigure}} или любой другой команды.

\section{Вызов команд}

Как уже было сказано, каждая команда начинается с \tex{} после
которого сразу, без пробелов, следует название команды. Например

\startdescr{code:} \hfill
\startTEX
ООО \quotation{Вектор}
\stopTEX
\stopdescr

\startdescr{result:} \hfill
\startCODERESULT
ООО \quotation{Вектор}
\stopCODERESULT
\stopdescr

В примере выше \quote{quotation} - это название команды.

У команд могут иметься еще и параметры. В примере выше у
команды \tex{quotation} есть один обязательный аргумент
(\quote{Вектор}) заключенный в фигурные скобки.

У команд бывают обязательные и необязательные параметры. Обязательные
параметры должны присутствовать всегда. Если вызвать команду без
обязательного параметра, то это приведет к ошибке компиляции
документа, о чем \c пояснит в сообщении.

Обязательные аргументы заключаются {\bf всегда в фигурные
скобки}. Обязательных параметров может быть больше одного, а может не
быть вообще. Ниже есть примеры использования
команды \tex{placefigure}, которая имеет два обязатльных
аргумента. Обязательный аргумент может быть пустым, но он должен быть
указан. Так, при вызове команды \tex{placefigure} первым обязательным
аргументом является подпись к картинке, которая может быть пустой

\vbox{
\startdescr{code:} \hfill
\startTEX
\placefigure[here,none]{}{      % первый
                                % обязательный
                                % аргумент пуст
  \externalfigure[example.jpg][width=2cm]
}
\stopTEX
\stopdescr
}

\vbox{
\startdescr{result:} \hfill
\startCODERESULT
\placefigure[here,none]{}{      % первый
                                % обязательный
                                % аргумент пуст
  \externalfigure[example.jpg][width=2cm]
}
\stopCODERESULT
\stopdescr
}

Обратите внимание, что первый обязательный аргумент присутствует,
просто он пустой. Пустые фигурные скобочки у \tex{placefigure} - и
есть первый обязательный аргумент.

Необязательные аргументы заключаются в {\bf квадратные скобки}. Их можно опускать вместе со скобочками. Вот так:

\vbox{
\startdescr{code:} \hfill
\startTEX
\placefigure{}{      % Необязательный аргумент
                     % опущен
  \externalfigure[example.jpg][width=2cm]
}
\stopTEX
\stopdescr
}

\vbox{
\startdescr{result:} \hfill
\startCODERESULT
\placefigure{}{      % Необязательный аргумент
                     % опущен
  \externalfigure[example.jpg][width=2cm]
}
\stopCODERESULT
\stopdescr
}

Дотошный читатель спросит, а почему тогда у \tex{externalfigure} два аргумента,
и оба не обязательных? Что и как будет размещать \tex{externalfigure}, если
вызвать ее без аргументов ?

\vbox{
\startdescr{code:} \hfill
\startTEX
\externalfigure
\stopTEX
\stopdescr
}

\vbox{
\startdescr{result:} \hfill
\startCODERESULT
\externalfigure
\stopCODERESULT
\stopdescr
}

Пожалуйста! В \c можно размещать картинки, не имея самих картинок. Просто вместо
картинки вставляется заглушка. Именно для этого у \tex{externalfigure} параметры
опциональные. К слову, если не указывать второй аргумент (размер
картинки), \tex{externalfigure} разместит картинку в ее натуральную величину,
без масштабирования.


\section{Размерности}

В некоторых командах есть параметры, значение которых измеряется в
расстоянии. Типовыми параметрами
являются \quote{width}, \quote{height}, \quote{offset} и подобные. Например,
если мы хотим задать конкретную ширину картинки, мы используем
параметр \quote{width}

\vbox{
\startdescr{code:} \hfill
\startTEX
\externalfigure[example.jpg][width=2cm]
\externalfigure[example.jpg][width=1in]
\externalfigure[example.jpg][width=0.2\textwidth]
\externalfigure[example.jpg][width=\dimexpr(\textwidth-15cm)]
\stopTEX
\stopdescr
}

\vbox{
\startdescr{result:} \hfill
\startCODERESULT
\externalfigure[example.jpg][width=2cm]
\externalfigure[example.jpg][width=1in]
\externalfigure[example.jpg][width=0.2\textwidth]
\externalfigure[example.jpg][width=\dimexpr(\textwidth-15cm)]
\stopCODERESULT
\stopdescr
}

В примере выше параметр \quote{width} задается в сантиметрах, дюймах, а
также, параметрически. \tex{textwidth} - это команда, возвращающая текущую
ширину текста.

В третьей строке ширина картинки задается как 0.2 от ширины текста. Если бы мы
хотели разместить картинку во всю ширину текста мы бы просто
написали \quote{width=\tex{textwidth}}

В четвертой строке используется команда \tex{dimexpr}, предназначенная для
вычисления выражений с размерами. Например, в \tex{dimexpr} можно сложить
сантиметры с дюймами, разделить ширину текста (или любое другое размерное
значение) на два и отнять один сантиметр. Обратите внимание, что выражение
заключается в {\bf круглые скобки} для команды \tex{dimexpr}, это такая
особенность данной команды.

Для указания едениц измерения используются постфиксы к цифрам

\placetable[here]{Еденицы измерения}{
\bTABLE
\bTR
\bTD Постфикс  \eTD
\bTD Назначение \eTD
\bTD Пример \eTD
\eTR

\bTR
\bTD mm \eTD
\bTD Миллиметры \eTD
\bTD 2mm \eTD
\eTR

\bTR
\bTD cm \eTD
\bTD Сантиметры \eTD
\bTD 5cm \eTD
\eTR

\bTR
\bTD em \eTD
\bTD Ширина буквы \quote{M} текущего шрифта \eTD
\bTD 1.2em \eTD
\eTR

\bTR
\bTD ex \eTD
\bTD Высота буквы \quote{x} текущего шрифта \eTD
\bTD 2ex \eTD
\eTR

\bTR
\bTD pt \eTD
\bTD Пункты. В пунктах измеряется высота шрифта. \eTD
\bTD 14pt \eTD
\eTR

\bTR
\bTD in \eTD
\bTD Дюймы \eTD
\bTD 3in \eTD
\eTR
\eTABLE}

Можно, например, задать расстояние так

\startTEX
\dimexpr(1in+2cm+1ex)
\stopTEX

Что будет означать один дюйм, два сантиметра, плюс высота символа \quote{x}
текущего шрифта.

\section{Перевод строки}

\c  не различает перевод строки от пробела в исходном тексте. Это означает, что если
вставить перевод строки куда либо, то на его месте, вероятнее всего, будет пробел

\vbox{
\startdescr{code:} \hfill
\startTEX
Мой дядя самых честных правил,
Когда не в шутку занемог,
Он уважать себя заставил
И лучше выдумать не мог.
\stopTEX
\stopdescr
}

\vbox{
\startdescr{result:} \hfill
\startCODERESULT
Мой дядя самых честных правил,
Когда не в шутку занемог,
Он уважать себя заставил
И лучше выдумать не мог.
\stopCODERESULT
\stopdescr
}

В примере выше вся ручная разметка проигнорировалась \c. Чтобы вручную вставить
перевод строки в нужное место используется команда \quote{\tex{\}}.

\vbox{
\startdescr{code:} \hfill
\startTEX
Мой дядя самых честных правил, \\
Когда не в шутку занемог, \\
Он уважать себя заставил \\
И лучше выдумать не мог. \\
\stopTEX
\stopdescr
}

\vbox{
\startdescr{result:} \hfill
\startCODERESULT
Мой дядя самых честных правил, \\
Когда не в шутку занемог, \\
Он уважать себя заставил \\
И лучше выдумать не мог. \\
\stopCODERESULT
\stopdescr
}

При этом, это все еще один параграф.

Любое количество пробелов рассматривается как один, нельзя заполнить пустое
место пробелами как в верде. Для вставки отвтупов в \c есть специальные средства.

\vbox{
\startdescr{code:} \hfill
\startTEX
раз             два
три

раз \hfill два

раз \hspace[big] два
\stopTEX
\stopdescr
}

\vbox{
\startdescr{result:} \hfill
\startCODERESULT
раз             два
три

раз \hfill два

раз \hspace[big] два
\stopCODERESULT
\stopdescr
}

\section{Вертикальный отступ}

Иногда нужно вставить вертикальный отступ. Это можно сделать
командами \tex{medskip}, \tex{bigskip}

\vbox{
\startdescr{code:} \hfill
\startTEX
параграф

\medskip
параграф

\bigskip
параграф
\stopTEX
\stopdescr
}

\vbox{
\startdescr{result:} \hfill
\startCODERESULT
параграф

\medskip
параграф

\bigskip
параграф
\stopCODERESULT
\stopdescr
}

Также можно использовать отступ конкретной величины с помощью команды \tex{blank}.

\vbox{
\startdescr{code:} \hfill
\startTEX
параграф

\blank[1.5cm]
параграф
\stopTEX
\stopdescr
}

\vbox{
\startdescr{result:} \hfill
\startCODERESULT
параграф

\blank[1.5cm]
параграф
\stopCODERESULT
\stopdescr
}

\vfill

Можно также приклеить верстку к самому низу страницы, если вставить
команду \tex{vfill}. \tex{vfill} автоматически заполняет пустое место, таким
образом, чтобы то, что идет после нее прилипло к концу страницы. Этот параграф
прижат книзу командой \tex{vfill}.

\page

\section{Горизонтальное выравнивание текста}

Выравнивание по всему документу настраивается с помощью \tex{setupalign} в
заголовке. Если нужно изменить вырванивание в конкретном месте, используются
команды \tex{startalignment} / \tex{stopalignment}.

\vbox{
\startdescr{code:} \hfill
\startTEX
Параграф с обычным выравниванием

\startalignment[middle]
Параграф выравнен по центру
\stopalignment

\startalignment[flushright]
Выравнено к правому краю
\stopalignment

\startalignment[width,stretch,nothyphenated]
С другой стороны дальнейшее развитие различных
форм деятельности способствует подготовки и
реализации системы обучения кадров, соответствует
насущным потребностям.
\stopalignment
\stopTEX
\stopdescr
}

\vbox{
\startdescr{result:} \hfill
\startCODERESULT
Параграф с обычным выравниванием

\startalignment[middle]
Параграф выравнен по центру
\stopalignment

\startalignment[flushright]
Выравнено к правому краю
\stopalignment

\startalignment[width,stretch,nothyphenated]
С другой стороны дальнейшее развитие различных
форм деятельности способствует подготовки и
реализации системы обучения кадров, соответствует
насущным потребностям.
\stopalignment
\stopCODERESULT
\stopdescr
}

В примере выше первый параграф сверстан с выравниванием по умолчанию. Второй
выравнен по центру. Третий к правому краю. Четвертый выровнен в ширину, при
этом, отключены переносы слов.

Вот краткий список параметров для \tex{startalignment}

\placetable[here]{Параметры для \tex{startalignment}}{
\bTABLE

\bTR
\bTD nothyphenated \eTD
\bTD Отключить переносы слов
\eTD

\bTR
\bTD stretch \eTD
\bTD Параметр разрешает \c растягивать текст больше обычного.
Опция бывает полезна для получения более красивого и ровного текста. \eTD
\eTR

\bTR
\bTD flushleft \eTD
\bTD Выравнивание по левому краю \eTD
\eTR

\bTR
\bTD flushright \eTD
\bTD Выравнивание по правому краю \eTD
\eTR

\bTR
\bTD width \eTD
\bTD Выравнивание по ширине \eTD
\eTR

\bTR
\bTD middle \eTD
\bTD Выравнивание по центру \eTD
\eTR

\bTR
\bTD tolerant \eTD
\bTD Опция разрешает \c делать более уродливые параграфы,
чтобы избежать выход текста за границы параграфа. Может быть
полезно при верстке таблиц. \eTD
\eTR

\eTABLE}

Больше подробностей в
\goto{документации}[url(http://wiki.contextgarden.net/Command/setupalign)]

\section{Принудительный разрыв страницы}

Обычно, \c сам разбивает текст на страницы, но иногда
требуется приказать \c начать новую страницу. Для этого есть
команда \tex{page}. Давайте попробуем ей воспользоваться

\vbox{
\startdescr{code:} \hfill
\startTEX
\page

Пожалуйста! Этот текст начался на новой странице,
так как перед ним была вставлена
команда \tex{page}.
\stopTEX
\stopdescr
}

\page

Пожалуйста! Этот текст начался на новой странице,
так как перед ним была вставлена
команда \tex{page}.

\section{Болдовый италик}

Чтобы менять параметры шрифта используются следующие
комбинации команд:

\placetable[here]{Команды изменения стилей}{
  \bTABLE
  \bTR
  \bTD \tex{tf} \eTD
  \bTD {\tf Нормальный шрифт} \eTD
  \eTR

  \bTR
  \bTD \tex{bf} \eTD
  \bTD {\bf Делает шрифт боловым} \eTD
  \eTR

  \bTR
  \bTD \tex{it} \eTD
  \bTD {\it Да, это включает италик} \eTD
  \eTR

  \bTR
  \bTD \tex{sl} \eTD
  \bTD {\sl Наклонный шрифт} \eTD
  \eTR

  \bTR
  \bTD \tex{bi} \eTD
  \bTD {\bi Болдовый италик !} \eTD
  \eTR
  \eTABLE
}

\placetable[here]{Размер шрифта}{
  \bTABLE

  \bTR
  \bTD \tex{tfxx} \eTD
  \bTD {\tfxx Совсем маленький шрифт} \eTD
  \eTR

  \bTR
  \bTD \tex{tfx} \eTD
  \bTD {\tfx Маленький шрифт} \eTD
  \eTR

  \bTR
  \bTD \tex{tf} \eTD
  \bTD {\tf Нормальный шрифт} \eTD
  \eTR

  \bTR
  \bTD \tex{tfa} \eTD
  \bTD {\tfa Большой шрифт} \eTD
  \eTR

  \bTR
  \bTD \tex{tfb} \eTD
  \bTD {\tfb Еще больше} \eTD
  \eTR

  \bTR
  \bTD \tex{tfc} \eTD
  \bTD {\tfc Совсем большой шрифт} \eTD
  \eTR

  \bTR
  \bTD \tex{tfd} \eTD
  \bTD {\tfd Ну совсем большой шрифт} \eTD
  \eTR

  \eTABLE
}

А теперь о том, как ими пользоваться. Каждая из этих команд
изменяет текущие параметры для шрифта до конца блока. Чтобы
применить наклонный шрифт только к части текста, нужно этот
текст поместить в блок. Блок создается при помощи фигурных
скобочек.

\vbox{
\startdescr{code:} \hfill
\startTEX

Тут у нас обычный шрифт, {\bf а этот кусок выделен жирным
  шрифтом}, опять обычный, {\it а теперь наклонный шрифт}
{\tfa \bf жинрый крупный шрифт} {\tfx \it мелкий наклонный
  шрифт}

\stopTEX
\stopdescr
}

\vbox{
\startdescr{result:} \hfill
\startCODERESULT
Тут у нас обычный шрифт, {\bf а этот кусок выделен жирным
  шрифтом}, опять обычный, {\it а теперь наклонный шрифт}
{\tfa \bf жинрый крупный шрифт} {\tfx \it мелкий наклонный
  шрифт}
\stopCODERESULT
\stopdescr
}

В примере выше фигурные скобочки не печатаются, потому, что
они используются для формирования блоков текста. Фигурная
скобка - это специальный символ, который обрабатывается \c
особым образом, чтобы напечатать фигурную скобку, нужно
ввести команду \quote{\tt \tex{}\{}. При выходе из блока
параметры шрифта сбрасываются на те, которые были перед
входом в блок. Блоки можно, также, делать вложенными:

\vbox{
\startdescr{code:} \hfill
\startTEX

Обычный текст {\bf пошел блок жирным {\tfa , пошел вложенный
    блок с большим размером шрифта, толщина шрифта снова
    нормальная, так как во вложенном блоке она не указана}
  вышли из вложенного блока, снова жирный шрифт} вышли из
блока, снова обычный шрифт.

\stopTEX
\stopdescr
}

\vbox{
\startdescr{result:} \hfill
\startCODERESULT

Обычный текст {\bf пошел блок жирным {\tfa , пошел вложенный
    блок с большим размером шрифта, толщина шрифта снова
    нормальная, так как во вложенном блоке она не указана}
  вышли из вложенного блока, снова жирный шрифт} вышли из
блока, снова обычный шрифт.

\stopCODERESULT
\stopdescr
}

\section{Цвет текста}

Цвет, как ни странно, задается командой \tex{color}, у нее
один обязательный аргумент - это текст, который будет
раскрашен. А также, один не обязатльный - цвет.

\vbox{
\startdescr{code:} \hfill
\startTEX

Иду я, как то, по лесу, и тут, \color[red]{из-за угла
  выворачивает танк} !

\stopTEX
\stopdescr
}

\vbox{
\startdescr{result:} \hfill
\startCODERESULT
Иду я, как то, по лесу, и тут, \color[red]{из-за угла
  выворачивает танк} !
\stopCODERESULT
\stopdescr
}

В качестве цвета указывается имя цвета, который должен быть
определен заранее, правда, есть и определенные заранее
цвета.

Подробнее в
\goto{документации}[url(http://wiki.contextgarden.net/Color)]

\section{Заголовки}

Для создания заголовков в \c есть специальный набор команд.

\placetable[here]{Команды для заголовков}{
  \bTABLE

  \bTR
  \bTD \tex{chapter\{\}} \eTD
  \bTD Определяет заголовок самого высокого уровня. \eTD
  \eTR

  \bTR
  \bTD \tex{section\{\}} \eTD
  \bTD Заголовок второго уровня \eTD
  \eTR

  \bTR
  \bTD \tex{subsection\{\}} \eTD
  \bTD Заголовок третьего уровня  \eTD
  \eTR

  \bTR
  \bTD \tex{subsubsection\{\}} \eTD
  \bTD Заголовок четвертого уровня \eTD
  \eTR

  \bTR
  \bTD[nr=2] ... \eTD
  \eTR
  \eTABLE}

У каждой из команд есть один обязательный аргумент - текст
заголовка. То есть, вызывается команда так:

\vbox{
\startdescr{code:} \hfill
\startTEX
\chapter{Источник плазменной резки Powermax65}
\stopTEX
\stopdescr
}

К сожалению, я не могу вставить эту команду на исполнение,
потому, что начнется новый раздел документа, и мне придется
рассказывать о источнике плазменной резки.

Внешний вид и поведение заголовков настраивается в преамбуле
документа. В качестве примера посмотрите этот документ, и
обратите внимание, что в нем нет висячих подзаголовков. То
есть, даже если на предыдущей странице достаточно места для
размещения там очередного подзаголовка с куском текста, но
этого места слишком мало, и подзаголовок будет выглядеть
лишним на той странице, то \c разместит его на следующей
странице автоматически. Если места на странице достаточно,
то \c просто разместит заголовок на той же странице. Вот так
это сделано в преамбуле:

\vbox{
\startdescr{code:} \hfill
\startTEX
\setuphead[chapter][align={nothyphenated,flushleft,tolerant},
  number=no,after={\blank[1.5em]}]
\setuphead[section][align={nothyphenated,flushleft,tolerant},
  style={\tfa \bf},number=no,before={\testpage[10] \blank[1em]},
  after={\blank[0.5em]}]
\stopTEX
\stopdescr
}

Итак, по порядку, \quote{align} - это настройка
выравнивания. Тут мы видим, что в заголовках запрещены
переносы и выходы за границы параграфа.

Далее \quote{number=no}, эта опция убирает номерацию
заголовков. Если поставить \quote{number=yes}, то
соответстующие заголовки будут пронумерованы.

\quote{before} - это команды, которые будут вставленны
непосредственно перед текстом заголовка автоматически. Тут
самое интересное - это настройки для \tex{section}, а
именно, команда \tex{testpage[10]}. Именно она выполняет всю
магию по размещению подзаголовков. Если на странице нельзя
разместить еще 10 строк обычного текста, эта команда
вставляет разрыв страницы. Таким образом, подзаголовок, если
вствлять его командой \tex{section}, никогда не будет висеть
в конце страницы, если на ней нет места минимум для 10 строк.

Также, может быть интересна опция \quote{style}, которая
задает внешний вид заголовка.

Более полная информация как всегда в
\goto{документации}[url(http://wiki.contextgarden.net/Titles)]

\section{Перечисления}

Для создания перечислений есть команды \tex{startitemize} /
\tex{stopitemize} и \tex{item}. Первые две нужны чтобы
определить область текста, являющуюся перечислением, а втора
чтобы вставить элемент списка. По умолчанию, каждый элемент
списка выделяется пулькой.

\vbox{
\startdescr{code:} \hfill
\startTEX
\startitemize
\item Это раз
\item Это два
\item Это три
\stopitemize
\stopTEX
\stopdescr
}

\vbox{
\startdescr{result:} \hfill
\startCODERESULT
\startitemize
\item Это раз
\item Это два
\item Это три
\stopitemize
\stopCODERESULT
\stopdescr
}

У \tex{startitemize} есть необязательные параметры.

\vbox{
\startdescr{code:} \hfill
\startTEX
\startitemize[n]
\item Это раз
\item Это два
\item Это три
\stopitemize
\stopTEX
\stopdescr
}

\vbox{
\startdescr{result:} \hfill
\startCODERESULT
\startitemize[n]
\item Это раз
\item Это два
\item Это три
\stopitemize
\stopCODERESULT
\stopdescr
}

Параметр \quote{n} заставляет \c вместо пулек вставлять
арабские цифры. Можно задать стартовый номер:

\vbox{
\startdescr{code:} \hfill
\startTEX
\startitemize[n,fit][start=42]
\item Это раз
\item Это два
\item Это три
\stopitemize
\stopTEX
\stopdescr
}

\vbox{
\startdescr{result:} \hfill
\startCODERESULT
\startitemize[n,fit][start=42]
\item Это раз
\item Это два
\item Это три
\stopitemize
\stopCODERESULT
\stopdescr
}

Отметьте, что \quote{start=42} - это второй необязательный
параметр. Также, появилась опция \quote{fit} в первом
параметре. \quote{fit} заставляет \c выделять место под
маркер списка более тщательно, учитывая содержимое маркера.

Можно сделать цифры римскими:

\vbox{
\startdescr{code:} \hfill
\startTEX
\startitemize[R,fit][start=42]
\item Это раз
\item Это два
\item Это три
\stopitemize
\stopTEX
\stopdescr
}

\vbox{
\startdescr{result:} \hfill
\startCODERESULT
\startitemize[R,fit][start=42]
\item Это раз
\item Это два
\item Это три
\stopitemize
\stopCODERESULT
\stopdescr
}

Списки, также, могут быть вложенными:

\vbox{
\startdescr{code:} \hfill
\startTEX
\startitemize
\item Это раз
  \startitemize
  \item Это раз раз
  \item Это раз два
  \item Это раз три
  \stopitemize
\item Это два
\item Это три
\stopitemize
\stopTEX
\stopdescr
}

\vbox{
\startdescr{result:} \hfill
\startCODERESULT
\startitemize
\item Это раз
  \startitemize
  \item Это раз раз
  \item Это раз два
  \item Это раз три
  \stopitemize
\item Это два
\item Это три
\stopitemize
\stopCODERESULT
\stopdescr
}

Вложенные списки имеют другой маркер, маркеры для каждой
глубины вложенности можно настроить, используя команду
\tex{setupitemize}.

Более подробно
\goto{как всегда}[url(http://wiki.contextgarden.net/Enumerations)]
