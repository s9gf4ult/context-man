\title{Общие правила}

Весь текст в \c разбит на параграфы. Параграфы при верстке отделаются
одной (и более) пустой строкой.

\startdescr{code:} \hfill
\startTEX
Параграф один.

Параграф два.


Параграф три.
\stopTEX
\stopdescr

\startdescr{result:} \hfill
\startCODERESULT
Параграф один.

Параграф два.


Параграф три.
\stopCODERESULT
\stopdescr



В примере выше три параграфа. \c игнорирует идущие подряд пустые строки, так
что, отбить расстояние энтерами как в верде не получиться. В данном случае
расстояние между параграфами после верстки будет одинаковое.

Также, в \c есть команды. Все команды начинаются с \tex{} (бэкслеш). Чтобы
напечатать сам бэкслеш нужно ввести команду ``\tex{tex}'' вот так

\startdescr{code:} \hfill
\startTEX
\tex{}
\stopTEX
\stopdescr

\startdescr{result:} \hfill
\startCODERESULT
\tex{}
\stopCODERESULT
\stopdescr

Кроме бэкслеша есть и другие специальные символы. С бэкслеша начинаются команды,
а с \% начинаются коментарии. Все что идет после знака коментария
игонорируется \c. Чтобы напечатать знак \%, а не коментарий
нужно \quote{оквочить} символ \% бэкслешем.

\startdescr{code:} \hfill
\startTEX
% Все, что идет после знака коментария игнорируется.
% Этот текст не будет виден после компиляции
Около 95\% опрошенных считают, что небо синего цвета.
\% все, что идет после знака процента видно
\stopTEX
\stopdescr

\startdescr{result:} \hfill
\startCODERESULT
% Все, что идет после знака коментария игнорируется.
% Этот текст не будет виден после компиляции
Около 95\% опрошенных считают, что небо синего цвета.
\% все, что идет после знака процента видно
\stopCODERESULT
\stopdescr

Кроме знака \% в \c есть и другие специальные символы.

\placetable[here]{Специальные символы}{
\bTABLE
\setupTABLE[column][3][width=4cm]
\bTR
\bTD Символ \eTD
\bTD Назначение \eTD
\bTD Как напечатать сам сивол \eTD
\eTR

\bTR
\bTD \% \eTD
\bTD Символ коментария. Все, что идет после него до конца строки игнорируется. \eTD
\bTD \tex{\%} \eTD
\eTR

\bTR
\bTD \tex{} \eTD
\bTD С этого символа начинаются команды \eTD
\bTD \tex{tex\{\}} \eTD
\eTR

\bTR
\bTD \$ \eTD
\bTD С этого символа начинаются математические формулы, им же и заканчиваются \eTD
\bTD \tex{\$} \eTD
\eTR

\bTR
\bTD \{ \} \eTD
\bTD Фигурные скобки используются для группирования текста и/или команд.
Также, в фигурных скобках передаются аргументы для команд. \eTD
\bTD \tex{\{} \tex{\}} \eTD
\eTR

\bTR
\bTD \# \eTD
\bTD Используется в формировании новых команд, обычно не нужен при простой верстке \eTD
\bTD \tex{\#} \eTD
\eTR

\eTABLE}

По сути, когда вы набираете \quote{\tex} с точки зрения \c ничем не отличается
от \quote{\tex{placefigure}} или любой другой команды.

\subject{Размерности}

В некоторых командах есть параметры, значение которых измеряется в
расстоянии. Типовыми параметрами
являются \quote{width}, \quote{heifht}, \quote{offset} и подобные. Например,
если мы хотим задать конкретную ширину картинки, мы используем
параметр \quote{width}

\startdescr{code:} \hfill
\startTEX
\externalfigure[example.jpg][width=2cm]
\externalfigure[example.jpg][width=1in]
\externalfigure[example.jpg][width=0.2\textwidth]
\externalfigure[example.jpg][width=\dimexpr(\textwidth-15cm)]
\stopTEX
\stopdescr

\startdescr{result:} \hfill
\startCODERESULT
\externalfigure[example.jpg][width=2cm]
\externalfigure[example.jpg][width=1in]
\externalfigure[example.jpg][width=0.2\textwidth]
\externalfigure[example.jpg][width=\dimexpr(\textwidth-15cm)]
\stopCODERESULT
\stopdescr

В примере выше параметр \quote{width} задается в сантиметрах, дюймах, а
также, параметрически. \tex{textwidth} - это команда, возвращающая текущую
ширину текста.

В третьей строке ширина картинки задается как 0.2 от ширины текста. Если бы мы
хотели разместить картинку во всю ширину текста мы бы просто
написали \quote{width=\tex{textwidth}}

В четвертой строке используется команда \tex{dimexpr}, предназначенная для
вычисления выражений с размерами. Например, в \tex{dimexpr} можно сложить
сантиметры с дюймами, разделить ширину текста (или любое другое размерное
значение) на два и отнять один сантиметр.

Для указания едениц измерения используются постфиксы к цифрам

\placetable[here]{Еденицы измерения}{
\bTABLE
\bTR
\bTD Постфикс  \eTD
\bTD Назначение \eTD
\bTD Пример \eTD
\eTR

\bTR
\bTD mm \eTD
\bTD Миллиметры \eTD
\bTD 2mm \eTD
\eTR

\bTR
\bTD cm \eTD
\bTD Сантиметры \eTD
\bTD 5cm \eTD
\eTR

\bTR
\bTD em \eTD
\bTD Ширина буквы \quote{M} текущего шрифта \eTD
\bTD 1.2em \eTD
\eTR

\bTR
\bTD ex \eTD
\bTD Высота буквы \quote{x} текущего шрифта \eTD
\bTD 2ex \eTD
\eTR

\bTR
\bTD pt \eTD
\bTD Пункты. В пунктах измеряется высота шрифта. \eTD
\bTD 14pt \eTD
\eTR

\bTR
\bTD in \eTD
\bTD Дюймы \eTD
\bTD 3in \eTD
\eTR
\eTABLE}

Можно, например, задать расстояние так

\startTEX
\dimexpr(1in+2cm+1ex)
\stopTEX

Что будет означать один дюйм, два сантиметра, плюс высота символа \quote{x}
текущего шрифта.

\subject{Перенос строки}

\c не различает перевод строки от пробела в исходном тексте. Это означает, что если
вставить перенос строки куда либо, то на его месте пероятнее всего будет пробел

\startdescr{code:} \hfill
\startTEX
Мой дядя самых честных правил,
Когда не в шутку занемог,
Он уважать себя заставил
И лучше выдумать не мог.
\stopTEX
\stopdescr

\startdescr{result:} \hfill
\startCODERESULT
Мой дядя самых честных правил,
Когда не в шутку занемог,
Он уважать себя заставил
И лучше выдумать не мог.
\stopCODERESULT
\stopdescr

В примере выше вся ручная разметка проигнорировалась \c. Чтобы вручную вставить
перевод строки в нужное место используется команда \tex{\}.


\startdescr{code:} \hfill
\startTEX
Мой дядя самых честных правил, \\
Когда не в шутку занемог, \\
Он уважать себя заставил \\
И лучше выдумать не мог. \\
\stopTEX
\stopdescr

\startdescr{result:} \hfill
\startCODERESULT
Мой дядя самых честных правил, \\
Когда не в шутку занемог, \\
Он уважать себя заставил \\
И лучше выдумать не мог. \\
\stopCODERESULT
\stopdescr

При этом, это все еще один параграф.

Любое количество пробелов рассматривается как один, нельзя заполнить пустое
место пробелами как в верде. Для вставки отвтупов в \c есть специальные средства.

\startdescr{code:} \hfill
\startTEX
раз             два
три

раз \hfill два

раз \hskip[5cm] два
\stopTEX
\stopdescr

\startdescr{result:} \hfill
\startCODERESULT
раз             два
три

раз \hfill два

раз \hspace[big] два
\stopCODERESULT
\stopdescr

\subject{Вертикальный отступ}

Иногда нужно вставить вертикальный отступ. Это можно сделать
командами \tex{medskip}, \tex{bigskip}

\startdescr{code:} \hfill
\startTEX
параграф

\medskip
параграф

\bigskip
параграф
\stopTEX
\stopdescr

\startdescr{result:} \hfill
\startCODERESULT
параграф

\medskip
параграф

\bigskip
параграф
\stopCODERESULT
\stopdescr

Также можно использовать отступ конкретной величины с помощью команды \tex{blank}.

\startdescr{code:} \hfill
\startTEX
параграф

\blank[1.5cm]
параграф
\stopTEX
\stopdescr

\startdescr{result:} \hfill
\startCODERESULT
параграф

\blank[1.5cm]
параграф
\stopCODERESULT
\stopdescr

\vfill

Можно также приклеить верстку к самому низу
страницы, если вставить
команду \tex{vfill}. \tex{vfill} автоматически
заполняет пустое место, таким образом, чтобы то,
что идет после нее прилипло к концу страницы. Этот
параграф прижат книзу командой \tex{vfill}.

\page  % \page чтобы следующий текст начинался со
       % следующей страницы.

\page
