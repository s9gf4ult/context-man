\mainlanguage[ru]

\usemodule  [simplefonts][size=12pt]
\usemodule [vim]
\defineframedtext [CODEBLOCK][background=color,backgroundcolor=lightgray,width=\dimexpr(\textwidth-1cm),
                              frame=off,offset=2mm]
\defineframedtext [CODERESULT][width=\dimexpr(\textwidth-1cm),frame=off,offset=2mm]

\definevimtyping[TEX][syntax=context,before=\startCODEBLOCK,after=\stopCODEBLOCK]

\usemodule  [metafun]

\setmainfont[dejavusans][expansion=quality,protrusion=quality]
\setmonofont[monaco][expansion=quality,protrusion=quality]
%% \setupinterlinespace[1.15em]
\setupalign[width,stretch]
\setuppagenumbering[location={footer,right}, style=bold]
\setupwhitespace[small]
\setupindenting[medium, yes]

\setuplayout[topspace=0mm,backspace=2.5cm,header=1cm,
             footer=1.5cm,width=\dimexpr(\paperwidth-4cm),
             height=\dimexpr(\paperheight-1cm),margin=1.5cm]

\setuphead[title][align={nothyphenated,flushleft,stretch}]
\setuphead[subject][align={nothyphenated,flushleft}]

\definedescription[descr][indenting=1cm,indentnext=yes,location=right,
  width=\textwidth,distance=\dimexpr(1cm-\textwidth),align={width,stretch}]

\setupTABLE[loffset=1mm,roffset=1mm,align={flushleft,stretch,lohi}]

\definemakeup
  [titlepage]
  [align=center,footerstate=none]

\def\c{ \hbox{\ConTeXt} }

\starttext

\subject{Общие правила}

Весь текст в \c разбит на параграфы. Параграфы при верстке отделаются
одной (и более) пустой строкой.

\startdescr{code:} \hfill
\startTEX
Параграф один.

Параграф два.


Параграф три.
\stopTEX
\stopdescr

\startdescr{result:} \hfill
\startCODERESULT
Параграф один.

Параграф два.


Параграф три.
\stopCODERESULT
\stopdescr



В примере выше три параграфа. \c игнорирует идущие подряд пустые строки, так
что, отбить расстояние энтерами как в верде не получиться. В данном случае
расстояние между параграфами после верстки будет одинаковое.

Также, в \c есть команды. Все команды начинаются с \tex{} (бэкслеш). Чтобы
напечатать сам бэкслеш нужно ввести команду ``\tex{tex}'' вот так

\startdescr{code:} \hfill
\startTEX
\tex{}
\stopTEX
\stopdescr

\startdescr{result:} \hfill
\startCODERESULT
\tex{}
\stopCODERESULT
\stopdescr

Кроме бэкслеша есть и другие специальные символы. С бэкслеша начинаются команды,
а с \% начинаются коментарии. Все что идет после знака коментария
игонорируется \c. Чтобы напечатать знак \%, а не коментарий
нужно \quote{оквочить} символ \% бэкслешем.

\startdescr{code:} \hfill
\startTEX
% Все, что идет после знака коментария игнорируется.
% Этот текст не будет виден после компиляции
Около 95\% опрошенных считают, что небо синего цвета.
\% все, что идет после знака процента видно
\stopTEX
\stopdescr

\startdescr{result:} \hfill
\startCODERESULT
% Все, что идет после знака коментария игнорируется.
% Этот текст не будет виден после компиляции
Около 95\% опрошенных считают, что небо синего цвета.
\% все, что идет после знака процента видно
\stopCODERESULT
\stopdescr

Кроме знака \% в \c есть и другие специальные символы.

\placetable[here]{Специальные символы}{
\bTABLE
\setupTABLE[column][3][width=4cm]
\bTR
\bTD Символ \eTD
\bTD Назначение \eTD
\bTD Как напечатать сам сивол \eTD
\eTR

\bTR
\bTD \% \eTD
\bTD Символ коментария. Все, что идет после него до конца строки игнорируется. \eTD
\bTD \tex{\%} \eTD
\eTR

\bTR
\bTD \tex{} \eTD
\bTD С этого символа начинаются команды \eTD
\bTD \tex{tex\{\}} \eTD
\eTR

\bTR
\bTD \$ \eTD
\bTD С этого символа начинаются математические формулы, им же и заканчиваются \eTD
\bTD \tex{\$} \eTD
\eTR

\bTR
\bTD \{ \} \eTD
\bTD Фигурные скобки используются для группирования текста и/или команд.
Также, в фигурных скобках передаются аргументы для команд. \eTD
\bTD \tex{\{} \tex{\}} \eTD
\eTR

\bTR
\bTD \# \eTD
\bTD Используется в формировании новых команд, обычно не нужен при простой верстке \eTD
\bTD \tex{\#} \eTD
\eTR

\eTABLE}

По сути, когда вы набираете \quote{\tex} с точки зрения \c ничем не отличается
от \quote{\tex{placefigure}} или любой другой команды.

\stoptext
